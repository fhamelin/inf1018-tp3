\documentclass[11pt,french]{article}

%--------------------
% basic packages
%--------------------
\usepackage{babel}
\usepackage[utf8]{inputenc}
\usepackage[T1]{fontenc}
\usepackage{graphicx}

% LIENS
\usepackage{hyperref}
\hypersetup{hidelinks, colorlinks, citecolor=black, linkcolor=black, urlcolor=blue}

% ENTÊTES
\usepackage{fancyhdr}
\pagestyle{fancy}
\lhead{INF1018\\Travail Pratique 3}
% STYLE SECTIONS
\usepackage{sectsty}
\sectionfont{\sectionrule{0ex}{0pt}{-1ex}{0.5pt}}

\usepackage{float}

\usepackage{array}
% CODE LISTING
\usepackage{listings}

\usepackage{fancyvrb}

\usepackage{amsmath}
\usepackage{amsthm}
\newtheorem{rem}{Remarque}

% CAPTIONS
%\usepackage{caption}
%\captionsetup[lstlisting]{ format=listing, labelfont=white, textfont=white, singlelinecheck=false, %margin=0pt, font={bf,footnotesize} }

% COULEURS
\usepackage{color}
\definecolor{mygreen}{rgb}{0,0.6,0}
\definecolor{mygray}{rgb}{0.5,0.5,0.5}
\definecolor{mymauve}{rgb}{0.58,0,0.82}

\lstset{ %
      backgroundcolor=\color{white},   % choose the background color; you must add \usepackage{color} or \usepackage{xcolor}
      basicstyle=\footnotesize,        % the size of the fonts that are used for the code
      breakatwhitespace=false,         % sets if automatic breaks should only happen at whitespace
      breaklines=true,                 % sets automatic line breaking
      captionpos=b,                    % sets the caption-position to bottom
      commentstyle=\color{mygreen},    % comment style
      deletekeywords={...},            % if you want to delete keywords from the given language
      escapeinside={\%*}{*)},          % if you want to add LaTeX within your code
      extendedchars=true,              % lets you use non-ASCII characters; for 8-bits encodings only, does not work with UTF-8
      frame=single,                    % adds a frame around the code
      keepspaces=true,                 % keeps spaces in text, useful for keeping indentation of code (possibly needs columns=flexible)
      keywordstyle=\color{blue},       % keyword style
      language=Octave,                 % the language of the code
      morekeywords={*,...},            % if you want to add more keywords to the set
      numbers=left,                    % where to put the line-numbers; possible values are (none, left, right)
      numbersep=5pt,                   % how far the line-numbers are from the code
      numberstyle=\tiny\color{mygray}, % the style that is used for the line-numbers
      rulecolor=\color{black},         % if not set, the frame-color may be changed on line-breaks within not-black text (e.g. comments (green here))
      showspaces=false,                % show spaces everywhere adding particular underscores; it overrides 'showstringspaces'
      showstringspaces=false,          % underline spaces within strings only
      showtabs=false,                  % show tabs within strings adding particular underscores
      stepnumber=2,                    % the step between two line-numbers. If it's 1, each line will be numbered
      stringstyle=\color{mymauve},     % string literal style
      tabsize=2,                       % sets default tabsize to 2 spaces
      title=\title                   % show the filename of files included with \lstinputlisting; also try caption instead of title
}
% Règle les problèmes d'encodage de caractères dans les fichiers sources
\lstset{literate=%
{á}{{\'a}}1 {é}{{\'e}}1 {í}{{\'i}}1 {ó}{{\'o}}1 {ú}{{\'u}}1
{Á}{{\'A}}1 {É}{{\'E}}1 {Í}{{\'I}}1 {Ó}{{\'O}}1 {Ú}{{\'U}}1
{à}{{\`a}}1 {è}{{\'e}}1 {ì}{{\`i}}1 {ò}{{\`o}}1 {ò}{{\`u}}1
{À}{{\`A}}1 {È}{{\'E}}1 {Ì}{{\`I}}1 {Ò}{{\`O}}1 {Ò}{{\`U}}1
{ä}{{\"a}}1 {ë}{{\"e}}1 {ï}{{\"i}}1 {ö}{{\"o}}1 {ü}{{\"u}}1
{Ä}{{\"A}}1 {Ë}{{\"E}}1 {Ï}{{\"I}}1 {Ö}{{\"O}}1 {Ü}{{\"U}}1
{â}{{\^a}}1 {ê}{{\^e}}1 {î}{{\^i}}1 {ô}{{\^o}}1 {û}{{\^u}}1
{Â}{{\^A}}1 {Ê}{{\^E}}1 {Î}{{\^I}}1 {Ô}{{\^O}}1 {Û}{{\^U}}1
{œ}{{\oe}}1 {Œ}{{\OE}}1 {æ}{{\ae}}1 {Æ}{{\AE}}1 {ß}{{\ss}}1
{ç}{{\c c}}1 {Ç}{{\c C}}1 {ø}{{\o}}1 {å}{{\r a}}1 {Å}{{\r A}}1
{€}{{\EUR}}1 {£}{{\pounds}}1
}


\begin{document}
    \begin{titlepage}
        \begin{center}
            \noindent\rule{13cm}{1pt}\\[0.4cm]
             % Titre
            \textsc{\huge \bfseries Travail Pratique 2}\\
                                    INF1010\\[0.4cm]
            \noindent\rule{13cm}{1pt}\\[5cm]

            % Auteurs
            \begin{minipage}{0.4\textwidth}
                \begin{flushleft}
                \large\emph{Auteur(s):}\\[0.5cm]
                    Simon \textsc{Désaulniers}\\
                    Frédéric \textsc{Hamelin}
                \end{flushleft}
            \end{minipage}
            \begin{minipage}{0.5\textwidth}
                \begin{flushright} \large
                    \emph{Professeur:} \\[0.5cm]
                    Mourad \textsc{Badri}, Ph.D
                    \vspace{\parskip}
                \end{flushright}
            \end{minipage}

            % On va au bas de la page
            \vfill
            {\large Université du Québec à Trois-Rivières\\ \today}
        \end{center}
        \thispagestyle{empty}
    \end{titlepage}

    \pagenumbering{arabic}
    \setcounter{page}{1}

    % INTRODUCTION ICI

    \section{JavaCC} % (fold)
    \label{sec:javacc}
        \subsection{Implémentation} % (fold)
        \label{sub:implementation}
            % - grammaire doit être écrite au complet
            % - Coder l'anlyseur syntaxique:
            % - nécessite d'apprendre la grammaire de JavaCC
            %
        % subsection implementation (end)
        Pour commencer, voici les conclusions que nous tirons de la facilité d'implémentation pour JavaCC.
        JavaCC sert à créer des analyseurs de grammaires et n'a donc pas d'analyseur de Java déjà fait à
        offrir. Il faut donc le programmer nous-même ou bien en trouver un que quelqu'un a fait. Il faut donc
        écrire la grammaire au complet et programmer l'analyseur syntaxique avant de pouvoir faire les manipulations
        voulues. Aussi, JavaCC utilise sa propre grammaire qui combine aussi du Java mais ça demande quand même
        l'apprentissage du nouveau langage.
        \subsection{Avantages et inconvénients} % (fold)
        \label{sub:avantages-inconvenients}
        % AVANTAGES
        % - flexible (il suffit de déclarer la grammaire)
        %   - permet d'analyser n'importe quel langage
        %   - permet de créer un arbre syntaxique soit-même.
        %   - exécution facile puisqu'aucune dépendence envers autres choses que Java

        % INCONVÉNIENTS
        % - Pour l'analyse d'un projet Java, JavaCC nécessite l'écriture de beaucoup de
        %   fonctionnalités déjà implémentées (comme dans AST).
        % - nécessite d'apprendre la grammaire de JavaCC
        %
        % subsection avantages-inconvenients (end)

        \subsection{Impact de l'évolution de Java} % (fold)
        \label{sub:impact-evolution-java}
        % - Impact significatif. Lorsque Java change, le programme doit changer.
        %
        % subsection impact-evolution-java (end)
        Étant donné que la grammaire et la syntaxe sont directement définis dans le code, si jamais une nouvelle version
        de Java arrive, on doit modifier le code pour rencontrer les nouveaux critères définis par la nouvelle version du
        langage.
    % section javacc (end)

    \section{AST d'Eclipse} % (fold)
    \label{sec:ast-eclipse}
        \subsection{Implémentation} % (fold)
        \label{sub:implementation}
        % - Un programmeur Java peut faciler créer un plugin
        %
        % subsection implementation (end)
        Pour ce qui a trait à la facilité d'implémentation, l'utilisation des AST d'Eclipse est mieux que l'utilisation de
        JavaCC puisqu'elle requiert seulement l'utilisation de code Java. C'est donc une tâche facile pour n'importe lequel
        programmeur Java.
        \subsection{Avantages et inconvénients} % (fold)
        \label{sub:avantages-inconvenients}
        % AVANTAGES
        % - grammaire de Java déjà analysée
        % - arbre syntaxique déjà créé

        % INCONVÉNIENTS
        % - exécution est confiné dans l'environnement Eclipse
        % - non-flexible au niveau du langage à analyser
        % subsection avantages-inconvenients (end)

        \subsection{Impact de l'évolution de Java} % (fold)
        \label{sub:impact-evolution-java}
        % AST rend les modifications de Java opaques. 
        Étant donné que les AST font l'analyse lexicale et syntaxique et créent l'arbre syntaxique automatiquement, 
        si jamais une nouvelle version de Java est mise sur le marché, il ne suffit que de compiler le code qu'on a
        déjà dans la nouvelle version et tout devrait fonctionner sans avoir eu à modifier le code que nous avons 
        déjà écrit.
        % subsection impact-evolution-java (end)
    % section ast (end)

\end{document}
